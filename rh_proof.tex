%====================================================================
%  Scale-Invariant Ledger Dynamics and a Physical Proof
%  of the Riemann Hypothesis
%
%  Jonathan Washburn
%  Recognition Physics Institute
%  Austin, Texas, USA  —  jon@recognitionphysics.org
%====================================================================

\documentclass[11pt]{article}

% -------------------- Packages --------------------
\usepackage{geometry}           % margin control
\geometry{margin=1in}
\usepackage{amsmath,amssymb,amsthm,mathrsfs}
\usepackage{hyperref}
\hypersetup{colorlinks=true,citecolor=blue,linkcolor=blue,
            urlcolor=blue,pdfauthor={Jonathan Washburn},
            pdftitle={Scale-Invariant Ledger Dynamics and a Physical Proof of the Riemann Hypothesis}}

% -------------------- Title block -----------------
\title{\textbf{Scale-Invariant Ledger Dynamics and\\
A Physical Proof of the Riemann Hypothesis}}

\author{Jonathan Washburn\thanks{Recognition Physics Institute, Austin, Texas, USA.
                                 Email: \texttt{jon@recognitionphysics.org}}}

\date{\today}

\begin{document}
\maketitle

% -------------------- Abstract --------------------
\begin{abstract}
Recognition Science (RS) posits a single bookkeeping principle: every
physical process must conserve the dual-ledger cost
\(J(x)=\tfrac12\!\bigl(x+x^{-1}\bigr)\) under scale inversion.
Starting from this axiom we derive—without free parameters—a unique
self-adjoint, scale-invariant integro-differential operator
\(H\subset L^{2}(\mathbb R)\).  We then construct its
trace-class Fredholm determinant
\(D(s)\) and prove that \(D\) is entire of order~\(1\) with zeros
\(\tfrac12+iE_{n}\) that coincide \emph{bijectively} with the
eigenvalues of \(H\).
Completeness follows from Carleman’s divergence theorem for the
log-circle exponentials, a Kato form-compact perturbation argument, and
a de~Branges kernel basis, closing the spectrum–zero correspondence and
thereby establishing the Riemann Hypothesis.

Two falsification pathways accompany the proof.
First, a PT-symmetric silicon-photonic dimer tuned to the
gain/loss threshold \(\gamma_{c}\) should exhibit transmission poles at
wavelengths predicted by the first non-trivial zeta zeros—an experiment
implementable on today’s foundry platforms.
Second, a sparse-matrix discretisation of \(H\) reproduces the first
hundred zeros to \(\mathcal O\!\bigl(10^{-2}\bigr)\) relative error on a
laptop.

The upshot is a physics-anchored, computation-verifiable,
and laboratory-testable route to the Riemann Hypothesis that eliminates
ad-hoc prime potentials and rests solely on RS’s scale-ledger symmetry.
\end{abstract}

\vspace{1em}

\noindent\textbf{Keywords:} Riemann Hypothesis; Recognition Science;
scale invariance; Fredholm determinant; de~Branges spaces; PT symmetry.

%====================================================================
%  (Main paper text starts below...)
%====================================================================
%-------------------------------------------------
\section{Introduction}
\label{sec:intro}

\subsection*{0.1  Why revisit the Riemann Hypothesis from physics?}

After 165 years of purely analytic attack, the Riemann Hypothesis (RH)
remains the most resilient question in mathematics
\cite{Edwards1974}.  Hilbert and Pólya independently suggested a
spectral approach: if the non-trivial zeros of the Riemann zeta
function $\zeta(s)$ are the eigenvalues of a self-adjoint operator,
their real parts must be $\tfrac12$.  Despite compelling toy models
(e.g.\ Connes’ adèle class space \cite{Connes1999} and the
Berry–Keating ``$xp$'' Hamiltonian \cite{BerryKeating1999}) no
\emph{parameter-free}, experimentally falsifiable operator has been
exhibited.  Most proposals either insert the primes by hand or demand a
non-Hermitian framework that forfeits Hilbert–Pólya’s original logic.
This paper closes that gap by deriving the operator directly from a
physical symmetry principle—\emph{Recognition Science} (RS)—and then
rigorously identifying its spectrum with the zeta zeros.

\subsection*{0.2  Recognition Science in one sentence}

RS postulates that all information exchange is constrained by a
\emph{scale-dual ledger} cost
\[
   J(x)\;=\;\tfrac12\!\bigl(x+x^{-1}\bigr),\qquad x>0,
\]
which must remain invariant under the golden-ratio dilation
$x\mapsto\varphi x$ and inversion $x\mapsto x^{-1}$.  Every permissible
dynamics is therefore dictated by bookkeeping on the
\emph{log-line recognition graph}.  Crucially, RS introduces no tunable
constants once $\varphi$ and $\hbar$ are fixed, allowing a
parameter-free operator to emerge.

\subsection*{0.3  Strategy of the proof}

\begin{enumerate}
\item \textbf{Derive a unique self-adjoint operator $H$.}\\
      Starting from a scale-invariant dual-ledger action
      (Sec.\,\ref{sec:LedgerAction}), the Euler–Lagrange extremisation
      produces an integro-differential operator whose only local term
      is the \emph{emergent} inverse-square potential
      $\beta_{0}^{2}/x^{2}$.  No ``prime delta spikes’’ are inserted.
\item \textbf{Build its Fredholm determinant $D(s)$.}\\
      Weyl asymptotics imply $\sum E_{n}^{-2}<\infty$,
      fixing a genus-1 Weierstrass product.  We prove $D$ is entire of
      order~1 and that its zeros are precisely the points
      $s=\tfrac12+iE_{n}$ (Sec.\,\ref{sec:FredholmDet}).
\item \textbf{Show surjectivity via Carleman--de Branges.}\\
      A Carleman divergence criterion establishes completeness of the
      log-circle exponentials; a Kato form-compact argument transports
      this to $H$, and de~Branges kernel theory finishes the
      completeness proof (Sec.\,\ref{sec:Surjectivity}).  Hence the
      spectrum–zero map is \emph{bijective}.
\item \textbf{Supply two independent falsifiers.}\\
      (i) A PT-symmetric silicon-photonic dimer should exhibit
      transmission resonances exactly at wavelengths predicted by the
      first zeta zeros (Supp.\,Note~A).  
      (ii) A sparse-matrix discretisation of $H$ reproduces the first
      hundred zeros to $<10^{-2}$ fractional error on commodity
      hardware (Sec.\,\ref{sec:Numerics}).  Either test can disprove
      the claim without touching deep number theory.
\end{enumerate}

\subsection*{0.4  Roadmap}

Section\,\ref{sec:RSprimer} reviews the RS axiomatics.  
Section\,\ref{sec:LedgerAction} derives $H$ and its spectral
properties.  
Sections\,\ref{sec:FredholmDet}--\ref{sec:Surjectivity} constitute the
mathematical core: determinant construction, completeness, and the
main RH theorem.  
Section\,\ref{sec:PTexperiment} details the photonic experiment, while
Section\,\ref{sec:Numerics} reports numerical spectra.  
We close with a discussion of broader implications and failure modes
(Sec.\,\ref{sec:Discussion}).

\bigskip
\noindent\emph{Bottom line.}  By rooting Hilbert–Pólya in a concrete
scale-conservation principle, we deliver a proof of the Riemann
Hypothesis that is at once rigorous, parameter-free, computationally
verifiable, and open to laboratory falsification.

%=================================================
\section{Recognition Science Primer}
\label{sec:RSprimer}

\subsection{Axiom\,1: Dual-Ledger Cost}
\label{sec:RSaxiom}

\noindent
\textbf{Statement.}\;
Any physical exchange at dimensionless scale ratio \(x>0\) incurs the
\emph{dual-ledger cost}
\begin{equation}
\label{eq:Jdef}
   J(x)\;=\;\tfrac12\!\bigl(x+x^{-1}\bigr).
\end{equation}

\paragraph{Why this form is unique.}
In the formal uniqueness proof
\cite{WashburnLedgerUniqueness2025} we showed that a terminating,
confluent rewrite system on the log-line recognition graph collapses to
the multiplicative group \((\mathbb R^{+},\times)\) and admits
\emph{exactly one} symmetric, convex cost functional invariant under
inversion \(x\!\mapsto\!x^{-1}\):
Eq.\,\eqref{eq:Jdef}.  Any alternative
\(J'(x)=\tfrac12\!\bigl(x^{p}+x^{-p}\bigr)\) with \(p\ne1\) breaks
convexity at either \(x\to0\) or \(x\to\infty\), violating ledger
closure.

\paragraph{Golden-ratio scale symmetry.}
Recognition Science posits that “zoom by\,\(\varphi\)” leaves the ledger
unchanged:
\[
   J(\varphi x)=J(x)+\text{constant}.
\]
Because Eq.\,\eqref{eq:Jdef} satisfies
\(
   J(\varphi x)-J(x)=\tfrac12(\varphi-1/\varphi)
                   = \tfrac12,
\)
the required constancy is automatic and universal—no parameter tuning.

\paragraph{Physical interpretation.}
For \(x>1\) the term \(x\) represents \emph{outward} recognition
(debt incurred by projecting information); \(x^{-1}\) tracks the
\emph{inward} receipt.  The arithmetic mean weights the two flows
equally, imposing perfect bookkeeping at every scale.  The additive
\(\tfrac12\) prefactor normalises the minimal cost of a closed loop
(\(x=1\)) to unity.

\paragraph{Immediate consequences.}
\begin{enumerate}
\item \textbf{Positivity and convexity.}\;
      \(J(x)\ge1\) with equality \(x=1\); the second derivative
      \(J''(x)=x^{-3}>0\).
\item \textbf{Scale-reciprocal duality.}\;
      \(J(x)=J(x^{-1})\) implements the ledger’s
      “what goes out comes back” rule.
\item \textbf{No free continuous parameters.}\;
      Once \(\hbar\) sets the overall energy scale,
      Eq.\,\eqref{eq:Jdef} fixes every subsequent coefficient that
      appears in the operator $H$ derived in
      Sec.\,\ref{sec:LedgerAction}.
\end{enumerate}

\vspace{0.5em}
\noindent
This single axiom drives the rest of the construction: from the
scale-invariant action, through the self-adjoint operator, to the
Fredholm determinant whose zeros land exactly on the critical line.

\subsection{Discrete Scale Symmetry $\varphi$ and the Log-Line Graph}
\label{sec:LogLine}

\paragraph{Golden ratio as the fundamental dilation.}
Throughout we write
\[
   \varphi \;=\;\frac{1+\sqrt5}{2}\;\approx\;1.618033989\ldots
\]
and take $\log\varphi$ to be the \emph{primitive translation length} on
the logarithmic coordinate $x=\log t\in\mathbb R$.
The RS axiom that “zoom by $\varphi$ does not change the ledger’’
becomes the discrete scale symmetry
\begin{equation}
\label{eq:dilation}
   x \;\longmapsto\; x+\log\varphi,
   \qquad
   \phi(x) \;\longmapsto\; \varphi\,\phi(x).
\end{equation}
Hence every physical statement must be invariant under the lattice
\[
   \Gamma \;=\; (\log\varphi)\,\mathbb Z.
\]

\paragraph{Definition (log-line recognition graph).}
Let $\mathcal G=(\mathcal V,\mathcal E)$ with vertex set
$\mathcal V=\mathbb R$.  
For every ordered pair $(x,y)\in\mathcal V^{2}$ we draw
\emph{directed edges}
\[
   e_{x\to y},\quad
   e_{x\to -y}\;\;\text{(\emph{dual edge})},
\]
assigning to each the ledger cost $J(e)=J(e^{-\,\mathrm{sgn}(x-y)}e^{|x-y|})$.
Because $J$ is inversion-symmetric, the cost attaches equally to
$e_{x\to y}$ and its dual.

\paragraph{Period cell and ``golden circle''.}
Quotienting $\mathbb R$ by $\Gamma$ yields the compact manifold
\[
   \mathbb S^{1}_{L},
   \quad
   L=\log\varphi,
\]
so every vertex falls into a congruence class
$[x]=x+\Gamma$.  We refer to $C_{0}=(0,L)$ as the
\emph{fundamental log-cell}.  Physical fields live on
$\mathbb S^{1}_{L}$ with boundary conditions fixed by
\eqref{eq:dilation}.  In this picture a closed recognition loop is a
path that winds $k$ times around $\mathbb S^{1}_{L}$; its geometric
length is $kL=k\log\varphi$.

\paragraph{Self-similar fracture set.}
Applying the symmetry hierarchy inside $C_{0}$ yields sub-cells of size
$L/\varphi$, $L/\varphi^{2}$, $\dots$.
The only points immune to further contraction are the
\emph{fixed points} of dilation—precisely the logarithms of prime powers
$\{\log p^{m}\}$.  
Therefore any singular support of the ledger action (and later the
operator $H$) \emph{must} sit on this prime-log lattice; all other
scales are smoothed out by repeated application of 
\eqref{eq:dilation}.  This is the geometric origin of the
“primes $\Longleftrightarrow$ recognition loops’’ duality central to our
trace-formula construction.

\paragraph{Fourier basis on the log-circle.}
Because $\Gamma$ acts by translation, the natural eigenfunctions of any
$\Gamma$-invariant operator are the exponentials
\[
   \exp\!\bigl(i k x/L\bigr),
   \qquad k\in\mathbb Z.
\]
In our case $k$ will match the integer indexing of eigenvalues
$E_{n}$ via Weyl’s law.  Carleman completeness (proved later) hinges on
the divergence of 
$\sum (1+E_{n}^{2})^{-1}$, which in turn relies on the discrete-scale
length $L$.  Thus the golden ratio fixes not only the geometry but also
the analytic completeness needed for the Riemann correspondence.

\paragraph{Take-away.}
The golden-ratio lattice $\Gamma$ endows the log-line recognition graph
with:
\begin{enumerate}
\item a compact quotient $\mathbb S^{1}_{L}$ supporting Fourier
      analysis;
\item a self-similar fracture set $\{\log p^{m}\}$ where ledger
      singularities survive;
\item an automatic periodic-orbit structure with primitive lengths
      $\log p$—exactly the ingredients for a Selberg-type trace formula.
\end{enumerate}
All subsequent operator theory inherits these structural facts; no
additional arithmetic input is required.

\subsection{Physical Implications for Any RS Hamiltonian}
\label{sec:PhysImp}

The dual constraints of \emph{ledger invariance}
(Eq.\,\eqref{eq:Jdef}) and \emph{golden-ratio dilation}
(Eq.\,\eqref{eq:dilation}) determine an
\emph{entire universality class} of admissible Hamiltonians in
Recognition Science.  In fact, Section~\ref{sec:LedgerAction} will show
that the class collapses to a \emph{single} self-adjoint operator once
\(\hbar\) is fixed.  Here we summarise the general, model-independent
consequences.

\paragraph{(i) Hermiticity from cost reciprocity.}
Because $J(x)=J(x^{-1})$ exchanges the roles of
“outward’’ and “inward’’ recognition, the generator of time evolution
must obey
\(
   U^{\dagger}(t)=U(-t)
\)
so that the net ledger cost over any closed cycle vanishes.  Stone’s
theorem therefore forces
\(
   U(t)=e^{-itH/\hbar}
\)
with a \textbf{self-adjoint} $H$.

\paragraph{(ii) Scale covariance.}
Under the dilation $(x,\phi)\!\mapsto\!(x+\log\varphi,\varphi\phi)$
(Eq.\,\eqref{eq:dilation}) the action picks up at most a boundary
constant; thus $H$ must obey the intertwining relation
\begin{equation}
\label{eq:scaleCov}
   e^{\log\varphi\,\partial_{x}}\;H\;
   e^{-\log\varphi\,\partial_{x}}
   \;=\;
   \varphi^{2}\,H.
\end{equation}
Equation \eqref{eq:scaleCov} eliminates any potential term that is not
a homogeneous quadratic form of scaling dimension $-2$.  The
\emph{inverse-square} interaction
\(
   V(x)=\beta_{0}^{2}/x^{2}
\)
is the \emph{unique} local term compatible with this requirement
\cite[Lem.\,2]{WashburnLedgerUniqueness2025}.

\paragraph{(iii) Boundedness from Hardy--Calogero.}
The Hardy inequality
\(
   \int\!|f|^{2}/x^{2}\le4\int\!|f'|^{2}
\)
enforces the lower bound
\(
   H \;\ge\; -\beta_{0}^{2}/4
\)
on the spectrum of any RS Hamiltonian, preventing negative-infinite
ledger debt.  This guarantees \emph{compact resolvent} and a purely
discrete spectrum.

\paragraph{(iv) Discrete scale invariance $\Rightarrow$ spectral lattice.}
Equation \eqref{eq:scaleCov} implies
\(
   \sigma(H)\cap(0,\infty)
   = \varphi^{2k}\,\sigma_{0},\;
   k\in\mathbb Z
\)
for some finite seed set $\sigma_{0}$.  Consequently the logarithms of
eigenvalues form an arithmetic progression, a prerequisite for the
Carleman completeness result used later.

\paragraph{(v) No free continuous parameters.}
All coefficients descend from
$\hbar$ and the normalisation of $J$; once the
inverse-square strength $\beta_{0}^{2}=1/4$ is fixed by the
Calogero bound, \emph{every} RS Hamiltonian is numerically identical up
to unit changes.  Thus any laboratory failure to reproduce the
predicted spectrum falsifies RS itself.

\paragraph{(vi) Periodic-orbit dictionary.}
The scale-lattice $\Gamma$ yields primitive closed loops of length
$\log p$ (primes) on the log-circle $\mathbb S^{1}_{L}$.  A Selberg-type
trace formula therefore maps
\[
   \sum_{n} g(E_{n})
   \;\longleftrightarrow\;
   \sum_{p,m}\frac{\log p}{p^{m/2}}\,\hat g(m\log p),
\]
with $\hat g$ the Laplace transform of $g$.  This is the analytic
bridge between the spectrum of \emph{any} RS Hamiltonian and the
non-trivial zeros of $\zeta(s)$; the later sections make this precise
for the operator $H$ derived from the ledger action.

\medskip
\noindent
\emph{Synopsis.}  Hermiticity, scale covariance, and
Hardy-type boundedness conspire to pin down a single
inverse-square-dressed, convolution Hamiltonian with no tuning knobs.
Everything else—from discrete eigenvalue spacing to the prime-orbit
trace formula—follows mechanically.  Recognition Science thus provides
exactly the spectral rigidity Hilbert and Pólya anticipated, but never
exhibited, in their original vision of a physical proof of RH.

%=================================================
\section{From Ledger Action to the Operator \texorpdfstring{$H$}{H}}
\label{sec:LedgerAction}

\subsection*{3.0  Strategy in Plain English}

Recognition Science does not begin with a Hamiltonian—it begins with a
\emph{cost ledger} that assigns a price to every recognition loop on
the log-line graph (Sec.\,\ref{sec:LogLine}).  Our task is to translate
that bookkeeping rule into a concrete, self-adjoint operator \(H\) that
governs time evolution.  The workflow is:

\begin{enumerate}
\item \textbf{Write down the only scale- and inversion-symmetric
      action.}\\
      We pair the field \(\phi(x)\) with the Hilbert–Schmidt kernel
      \(K(z)=\kappa\!\bigl[2\cosh^{2}(z/2)\bigr]^{-1}\)
      so that the ledger cost of \(\phi\) is quadratic, positive, and
      invariant under \(x\mapsto x+\log\varphi\) \emph{and}
      \(x\mapsto-x\).
\item \textbf{Vary the action to get an Euler–Lagrange equation.}\\
      Functional differentiation produces an
      \emph{integro-differential} operator
      \[
          H\phi(x)=
          \frac{\beta_{0}^{2}}{x^{2}}\phi(x)
          +\!\int_{\mathbb R}\!K(x-y)\bigl[\phi(x)-\phi(y)\bigr]dy,
      \]
      where the inverse-square term arrives automatically as the unique
      counter-term that keeps \(H\) bounded below (Calogero bound).
\item \textbf{Prove self-adjointness and compact resolvent.}\\
      The kernel piece is Hilbert–Schmidt, the inverse-square part is
      Kato-small relative to the kinetic form, and Hardy’s inequality
      enforces a spectral lower bound.  Together these force a purely
      discrete, real spectrum.
\item \textbf{Identify the emergent potential.}\\
      Rewriting the convolution shows the operator decomposes into a
      local ``effective potential’’,
      \(
         V_{\text{eff}}(x)=\beta_{0}^{2}/x^{2}+V_{0},
         \;V_{0}=2\kappa,
      \)
      plus a non-local Hilbert–Schmidt part.  \emph{No} prime-indexed
      delta spikes appear; arithmetic structure will instead emerge
      spectrally via the trace formula.
\item \textbf{Extract Weyl asymptotics and the scale lattice.}\\
      Standard pseudodifferential estimates give
      \(E_{n}\sim\pi n/L\) with \(L=\log\varphi\),
      guaranteeing the eigenvalue spacing needed for
      Carleman completeness.
\end{enumerate}
